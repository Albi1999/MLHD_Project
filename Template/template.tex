\documentclass[10pt, conference, letterpaper]{IEEEtran}

\usepackage{algorithm}
\usepackage{algorithmicx}
\usepackage{algpseudocode}
\usepackage{amsfonts}
\usepackage{amsmath}
\usepackage{amssymb}
\usepackage[ansinew]{inputenc} 
\usepackage{xcolor}
\usepackage{mathtools}
\usepackage{graphicx}
\usepackage{caption}
\usepackage{subcaption}
\usepackage{import}
\usepackage{multirow}
\usepackage{cite}
\usepackage[export]{adjustbox}
\usepackage{breqn}
\usepackage{mathrsfs}
\usepackage{acronym}
%\usepackage[keeplastbox]{flushend}
\usepackage{setspace}
\usepackage{bm}
\usepackage{stackengine}

\usepackage{listings}

\lstset{%
 backgroundcolor=\color[gray]{.85},
 basicstyle=\small\ttfamily,
 breaklines = true,
 keywordstyle=\color{red!75},
 columns=fullflexible,
}%

\lstdefinelanguage{BibTeX}
  {keywords={%
      @article,@book,@collectedbook,@conference,@electronic,@ieeetranbstctl,%
      @inbook,@incollectedbook,@incollection,@injournal,@inproceedings,%
      @manual,@mastersthesis,@misc,@patent,@periodical,@phdthesis,@preamble,%
      @proceedings,@standard,@string,@techreport,@unpublished%
      },
   comment=[l][\itshape]{@comment},
   sensitive=false,
  }

\usepackage{listings}

% listings settings from classicthesis package by
% Andr\'{e} Miede
\lstset{language=[LaTeX]Tex,%C++,
    keywordstyle=\color{RoyalBlue},%\bfseries,
    basicstyle=\small\ttfamily,
    %identifierstyle=\color{NavyBlue},
    commentstyle=\color{Green}\ttfamily,
    stringstyle=\rmfamily,
    numbers=none,%left,%
    numberstyle=\scriptsize,%\tiny
    stepnumber=5,
    numbersep=8pt,
    showstringspaces=false,
    breaklines=true,
    frameround=ftff,
    frame=single
    %frame=L
}

\renewcommand{\thetable}{\arabic{table}}
\renewcommand{\thesubtable}{\alph{subtable}}

\DeclareMathOperator*{\argmin}{arg\,min}
\DeclareMathOperator*{\argmax}{arg\,max}

\def\delequal{\mathrel{\ensurestackMath{\stackon[1pt]{=}{\scriptscriptstyle\Delta}}}}

\graphicspath{{./figures/}}
\setlength{\belowcaptionskip}{0mm}
\setlength{\textfloatsep}{8pt}

\newcommand{\eq}[1]{Eq.~\eqref{#1}}
\newcommand{\fig}[1]{Fig.~\ref{#1}}
\newcommand{\tab}[1]{Tab.~\ref{#1}}
\newcommand{\secref}[1]{Section~\ref{#1}}

\newcommand\MR[1]{\textcolor{blue}{#1}}
\newcommand\red[1]{\textcolor{red}{#1}}
\newcommand{\mytexttilde}{{\raise.17ex\hbox{$\scriptstyle\mathtt{\sim}$}}}

%\renewcommand{\baselinestretch}{0.98}
% \renewcommand{\bottomfraction}{0.8}
% \setlength{\abovecaptionskip}{0pt}
\setlength{\columnsep}{0.2in}

% \IEEEoverridecommandlockouts\IEEEpubid{\makebox[\columnwidth]{PUT COPYRIGHT NOTICE HERE \hfill} \hspace{\columnsep}\makebox[\columnwidth]{ }} 

\title{``We Rock the Hizzle and Stuff'' \\ hints on how to write a nice research essay}

\author{Michele Rossi$^\dag$, Author two$^\ddag$
\thanks{$^\dag$Department of Information Engineering, University of Padova, email: \{rossi\}@dei.unipd.it}
\thanks{$^\ddag$Author two affiliation, email: \{name.surname\}@dei.unipd.it}
\thanks{Special thanks / acknowledgement go here.}
} 

\IEEEoverridecommandlockouts

\newcounter{remark}[section]
\newenvironment{remark}[1][]{\refstepcounter{remark}\par\medskip
   \textbf{Remark~\thesection.\theremark. #1} \rmfamily}{\medskip}

\begin{document}

\maketitle

\begin{abstract}
Future vehicular communication networks call for new solutions to support their capacity demands, by leveraging the potential of the \mbox{millimeter-wave} (\mbox{mm-wave}) spectrum. Mobility, in particular, poses severe challenges in their design, and as such shall be accounted for. A key question in \mbox{mm-wave} vehicular networks is how to optimize the \mbox{trade-off} between directive Data Transmission (DT) and directional Beam Training (BT), which enables it. In this paper, learning tools are investigated to optimize this \mbox{trade-off}. In the proposed scenario, a Base Station (BS) uses BT to establish a \mbox{mm-wave} directive link towards a Mobile User (MU) moving along a road. To control the BT/DT \mbox{trade-off}, a Partially Observable (PO) Markov Decision Process (MDP) is formulated, where the system state corresponds to the position of the MU within the road link. The goal is to maximize the number of bits delivered by the BS to the MU over the communication session, under a power constraint. The resulting optimal policies reveal that adaptive BT/DT procedures significantly outperform \mbox{common-sense} heuristic schemes, and that specific mobility features, such as user position estimates, can be effectively used to enhance the overall system performance and optimize the available system resources.\\ 

\MR{This is an example abstract. It is $204$ words long, I would say an abstract should not be longer than $250$ words and some Transactions journals of the IEEE are currently putting a strict limit of $200$ words. Here, you should briefly state: 
\begin{enumerate}
\item the technical scenario/field of research and its timeliness/relevance in general (one sentence),
\item what you do in the report/paper and why it is important, how it advances the state of the art in its field (a few sentences), 
\item summarize the main and best results of your study/proposal/method (one or two sentences),
\item (optional) how others could benefit from your results for further research, or within commercial products (one sentence).
\end{enumerate}  
The abstract is one of the most important parts of the paper/report. You have roughly one minute to catch the reader's attention. A poor abstract may already move you towards the rejection side in the reviewer's decision process. In the abstract, 1) establish the context, 2) motivate the problem, 3) briefly describe the solution, and 4) present the main results of your work. Ideally, use one (short) sentence for each of the previously mentioned items to keep your abstract short. Overall, this should be a short summary of the whole content of your paper, including your results.}\\

\MR{See the abstract as a personal challenge for each of your papers. Finally, the abstract should contain the main message about your work, so that the reader will now what she/he can find even without reading it (as it is the case most of the times). The abstract is a mini-paper on its own and, as such, it is a major endeavor to write.}\\ 

\red{I suggest to write the Abstract as the very last thing. You may sketch it at the beginning, but then always finalize it at the end.}
\end{abstract}

\IEEEkeywords
Self Organizing Maps, Unsupervised Learning, Optimization, Neural Networks, Recurrent Neural Networks. \MR{A list of keywords defining the tools and the scenario. I would not go beyond {\it six} keywords.}
\endIEEEkeywords


% !TEX root = template.tex

\section{Introduction}
\label{sec:introduction}

\begin{remark}
\textbf{Paper contribution:} First and foremost decide on what precisely is the contribution of your paper over the state of the art. If you think you have several contributions, {\it focus on the most important one}. It may be that you can add one or two contributions as side topics, but in general you should focus on the most important one so as {\it to keep your paper focused}. As a side note: If you think you have several contributions for a single paper, you should probably invest in researching state of the art and related work more thoroughly. When you know your contribution, think of a good title.
\end{remark}

\begin{remark} 
\textbf{Title:} Find a short and precise title for you paper exactly matching the content. It's worth investing time into this matter, as {\it the title will be that part of the paper by which it will be referenced} (in case it gets published).
\end{remark}

\begin{remark}
\textbf{How to organize your writeup with Latex:} I find it useful, especially for a paper with multiple authors, to split your manuscript into multiple source Latex files. This is very easily accomplished by having single {\it root} file, e.g., \texttt{main.tex}, where you define the article class, all the \mbox{user-defined} commands, the paper formatting options (e.g., one vs two columns, the page margins, the text size and the fonts, etc.). For this project, the root is called \texttt{template.tex}. From the root you then call, in sequential order, a number of other latex source files through the command \texttt{\textbackslash input\{filename.tex\}}. It is often convenient to have one of such files for each section of your paper. This facilitates editing multiple sections in parallel and keeping your project synchronized with some versioning and revision control system such as \texttt{svn} or \texttt{github} (highly recommended).
\end{remark}

\begin{remark}
\textbf{Compiling trick:} For each source Latex file that you call from the main, I recommend you copy the following command in the very first line\\
\texttt{\% !TEX root = template.tex}\\
where \texttt{template.tex} is the name of the root Latex file. This command tells the Latex compiler that your main root file is \texttt{template.tex}, and it makes it possible to compile the paper from any of the sub-files. Example: imagine you are editing \texttt{introduction.tex} and have that open in a window of your preferred Latex editor. With this command, you can compile from the local \texttt{introduction.tex} window, without having to open and switch every time to the root file window and compile from it. This will save you hours.  
\end{remark}

\red{Maximum length for the whole report is 9 pages. Abstract, introduction and related work should take max two pages.}\\

\noindent \textbf{Recommended structure for the intro:} you may use the following structure. 
\begin{itemize}
\item \textbf{General (short) intro:} One paragraph to introduce your work, describing the scenario {\it at large}, its relevance, to prepare the reader to what follows and convince her/him that the paper focuses on an important setup/problem. Please keep this part short (I usually do five to eight lines), as this part is rather standard, \textbf{but} at the same time it has to be there. 
\item \textbf{Put the problem into perspective:} A second paragraph where you immediately delve into the specific problem that you are tackling, starting to detail your contribution. Here, you describe the importance of such problem, providing examples (citing papers from the literature, possibly recent ones) of previous solutions attempts, and of why these failed {\it to provide a complete answer}. This second paragraph should not be too long, as otherwise the reader will get bored and will abandon your paper... It should be concisely written, something like 5 to 10 lines.
\item \textbf{Present the paper contribution:} A third paragraph were you state what you do in the paper, this should also be concisely written. A good rule of thumb is to make it max 10/15 lines. Here, you should state up front:
\begin{enumerate}
\item \textbf{problem}: the problem at stake, 
\item \textbf{relevance}: the relevance and timeliness of what you propose, 
\item \textbf{approach}: the technique/approach you use, possibly underlying its novelty, efficiency, 
\item \textbf{value}: underline the value/novelty of your proposal referencing (recent) papers from the literature,
\item \textbf{applicability}: tell the reader how she/he can take advantage of your work, e.g., how your work/results can be reused/exploited to achieve further scientific, technical or practical (integrated into products?) goals.
\end{enumerate}
\item \textbf{Summary of contributions:} After this, you may want to provide an itemized list to summarize the paper contributions. Rule of thumb: from three to six items, from three to four lines each.
\item \textbf{Closing (paper structure):} You finish up by detailing the paper structure, this should be three to four lines. It is customary to do so, although I admit it may be of little use. It usually goes like: {\it ``This report (paper) is structured as follows. In Section II we describe the state of the art, the system and data models are respectively presented in Sections~III and~IV. The proposed signal processing technique is detailed in Section~V and its performance evaluation is carried out in Section~VI. Concluding remarks are provided in Section~VII.''}
\end{itemize}

\begin{remark} 
Lately, I tend to write introduction plus abstract within a single page. This forces me to focus on the important messages that I want to deliver about the paper, leaving out all the ``blah blah''. \textbf{Remember:} 1) {\it less is more}, 2) writing a compact ({\it snappy}) piece of technical text is much more difficult than writing lengthy stuff with no space constraints.
\end{remark}


% !TEX root = template.tex

\section{Related Work}
\label{sec:related_work}

\noindent \textbf{Some hints:} 
\begin{itemize}
\item \textbf{Goal:} The goal of this section is to describe what has been done so far in {\it the} literature. You should focus on and briefly describe the work done in the best papers that you have read. 
\item \textbf{Length:} One full column is fine but often this takes one column and a half. It is very easy to use a full page, although this may just due to your sloppiness... if you carefully go through the one page long version, you often find it possible to compact it in one column and a half. In any event, I would make this section no longer than one page, this leads to an overall {\it two pages} including abstract, introduction and related work. I believe this is a fair amount of space in most cases.
\item \textbf{Approach:} For each you should comment on the paper's contribution, on the good and important findings of such paper and also, \textbf{1)} on why these findings are not enough and \textbf{2)} how these findings are improved upon/extended by the work that you present here. At the end of the section, you may recap the main paper contributions (maybe one or two, the most important ones) and how these extend/improve upon previous work.
\end{itemize}
\begin{itemize}
\item \textbf{References:} please follow this {\it religiously}. It will help you a lot. Use the Latex \texttt{Bibtex} tool to manage the bibliography. A \texttt{Bibtex} example file, named \texttt{biblio.bib} is provided with this template.\\

\item \textbf{Citing conference/workshop papers:} I recommend to always include the following information into the corresponding \texttt{bibitem} entry: 
\begin{enumerate}
\item author names, 
\item paper title, 
\item conference / workshop name, 
\item conference / workshop address, 
\item month, 
\item year.
\end{enumerate}
Examples of this are: \cite{Zargham-2011}\cite{Sadler-2006}.\\

\item \textbf{Citing journal papers:} I recommend to always include the following information into the corresponding \texttt{bibitem} entry: 
\begin{enumerate}
 \item author names, 
 \item paper title, 
 \item full journal name, 
 \item volume (if available), 
 \item number (if available), 
 \item month, 
 \item pages (if available), 
 \item year. 
 \end{enumerate}
 Examples of this are: \cite{Shannon-1948}\cite{Boyd-2011}\cite{Zordan-2014}.\\

\item \textbf{Citing books:} I recommend to always include the following information into the corresponding \texttt{bibitem} entry: 
\begin{enumerate}
\item author names, 
\item book title, 
\item editor, 
\item edition, 
\item year.
\end{enumerate}
\end{itemize}
%

\begin{remark}
Note that some of the above fields may not be shown when you compile the Latex file, but this depends on the bibliography settings (dictated by the specific Latex style that you load at the beginning of the document). You may decide to include additional pieces of information in a given bibliographic entry, but please, \textbf{be consistent} across all the entries, i.e., use the same fields for the same publication type. Note that some of the fields may not be available (e.g., the paper {\it volume}, {\it number} or the {\it pages}).
\end{remark}

% !TEX root = template.tex

\section{Processing Pipeline}
\label{sec:processing_architecture}

\begin{remark}
\textbf{On tailoring the paper structure to your needs:} The structure recommended for the previous sections is rather standard and could work for different papers with differing technical content, the structure and the paper content from here on highly depends on the type of paper, possibilities are: mostly based on theoretical analysis, showing experimental design/activity, proposing a new technique and analyzing its performance via experiments or simulation. For the HDA course we deal with machine learning and, in detail, with training and testing neural network architectures to perform some specific inference or classification task. The following structure and comments are specifically addressing this type of technical content.
\end{remark}

\begin{remark}
\textbf{Why having this section:} With this section, we start the technical description with a {\it high level} introduction of your work (e.g., processing pipeline). Here, you do not have to necessarily go into the technical details of every block/algorithm of your design, this will be done later as the paper develops. What I would like to see here is a high level description of the approach, i.e., which processing blocks you used, what they do (in words) and how these were combined, etc. This section should introduce the reader to your design, explain the different parts/blocks, how they interact and why. You should not delve into technical details for each block, but you should rather explain the big picture. \MR{Besides a well written explanation, I often use a nice diagram containing the various blocks and detailing their interrelations.}
\end{remark} 

\noindent \textbf{Writing tips:} Sections, Figures and Tables are usually shortened as Sec., Fig., Tab. 
\begin{itemize}
\item \textbf{Cross referencing:} In Latex, cross referencing is easy. You need to label an object through the \texttt{\textbackslash label\{labelid\}} command and referencing it where you need it through the \texttt{\textbackslash ref\{labelid\}} command. 
\item \textbf{Suggestion:} I suggest to cross reference a table using \texttt{Tab.\mytexttilde\textbackslash ref\{tab:tableid\}}, the same holds for figures and sections, by just replacing ``\texttt{Tab.}'' with ``\texttt{Fig.}'' and ``\texttt{Sec.}''. Of course when defining the table, you need to add a Latex command \texttt{\textbackslash label\{tab:tableid\}} in the right place inside the table Latex environment to \mbox{cross-link} it. The tag \texttt{tab:tableid} is user defined and is the identifier that you associate with the table in question. You could call it \texttt{pippo} of you wish, but I recommend to use something like ``\texttt{tab:tableid}'' for tables, ``\texttt{eq:eqid}'' for equations, ``\texttt{sec:secid}'' for sections and so forth, where \texttt{tableid} has to be unique for each table in the document. The same applies to figures and all other objects. I guess you got the idea. This will lead to a neat Latex code and will facilitate cross referencing while avoiding duplicate labels, especially in large Latex documents (think of a book for example). 
\item \textbf{But what about the tilde?} This is a nice trick I have learned from a friend (many years ago from Prof. Frank Fitzek, now at TU Dresden). When you write \texttt{Tab.\mytexttilde\textbackslash ref\{tab:tableid\}} Latex knows that \texttt{Tab.} and the corresponding table number \texttt{\textbackslash ref\{tab:tableid\}} must be displayed within the same line, i.e., they can never be broken across lines. This is nice and desirable I believe. I always use it for all referenced material, including citations; example: ``As done in\texttt{\mytexttilde\textbackslash cite\{suppa-wu-2019\}}.''
\item \textbf{More about breaking stuff across lines} often times you have composed words, in line equations, etc. and for some reason you would like Latex to never break them across lines. Example: the Latex command \texttt{\textbackslash mbox\{Neural Networks (NN)\}} is processed by Latex so that ``\mbox{Neural Networks (NN)}'' is never broken across lines. I use this very often, also for inline equations that I do not want Latex to split across subsequent lines.
\end{itemize}

\section{Signals and Features}
\label{sec:model}

Being a machine learning paper, I would have here a section describing the signals you have been working on. If possible, you should describe, in order, 
\begin{enumerate}
\item the measurement setup (if relevant), how input data is formatted (e.g., vectors of fixed size measured at regular sampling times), 
\item how the signals were \mbox{pre-processed}, e.g., to remove noise, artifacts, fill gaps (missing points) or to re-interpolate the signals to represent them through a constant sampling rate, etc.
\item after this, you should describe how {\it feature vectors} were obtained from the \mbox{pre-processed} signals. If signals are {\it time series} this also implies stating the segmentation / windowing strategy that you adopted, to then describe how you obtained a feature vector for each time window. Also, if you use existing feature extraction approaches, you may want to briefly describe them as well, in addition to (and before) your own (possibly new) feature extraction method.
\end{enumerate}

\MR{Last but not least, this section should also contain information on how you have split the dataset into training, validation and test sets. This will be briefly recalled within the ``Results'' section.}

\section{Learning Framework}
\label{sec:learning_framework}

Here you finally describe the learning strategy / algorithm that you conceived and used to solve the problem at stake. A good diagram to exemplify how learning is carried out is often very useful. In this section, you should describe the learning model, its parameters, any optimization over a given parameter set, etc. You can organize this section into \mbox{sub-sections}. You are free to choose the most appropriate structure.

\begin{remark}
Note that the diagram that you put here differs from that of Section~\ref{sec:processing_architecture} as here you show the details of how your learning framework, or the core of it, is built. In Section~\ref{sec:processing_architecture} you instead provide a high-level description of the involved processing blocks, i.e., you describe the {\it processing flow} and the rationale behind it.
\end{remark}

\noindent \textbf{On math typesetting:} there are many Latex tricks that you should use to produce a high quality technical essay. A few are listed below, in random order:
\begin{itemize}
\item \textbf{Vectors and matrices:} $x$ is a scalar, whereas $\bm{x}$ (in bold) is a vector, and $\bm{X}$ is a matrix with elements $\bm{X} = [x_{ij}]$. For bold symbols you may use the \texttt{\textbackslash bm} Latex command, e.g., \texttt{\textbackslash bm(x)}.
\item \textbf{Operators:} such as $\max$, $\min$, $\arg\!\max$, $\arg\!\min$ and special functions such as $\log(\cdot)$, $\exp(\cdot)$, $\sin(\cdot)$, $\cos(\cdot)$ are obtained through specific latex commands \texttt{\textbackslash min}, \texttt{\textbackslash max}, \texttt{\textbackslash arg\textbackslash!min}, \texttt{\textbackslash arg\textbackslash!max}, \texttt{\textbackslash log}, \texttt{\textbackslash exp}, \texttt{\textbackslash sin}, \texttt{\textbackslash cos}, etc. Use them! $log$, $exp$, $min$, $sin$, $cos$, etc., look ugly.
\item \textbf{Sets} can be represented through calligraphic fonts, e.g., $\mathcal{S}$, $\mathcal{F}$, $\mathcal{B}$, etc., obtained using the Latex command \texttt{\textbackslash mathcal{\{S\}}}, etc.
\item \textbf{Equations:} for a single equation use the \texttt{equation} Latex environment. Example: 
\begin{equation}
\label{eq:sigmoid}
\sigma(x) = \frac{1}{1+e^{-x}}.
\end{equation} 
Now, using round brackets $($ and $)$ we get
\begin{equation}
\label{eq:sigmoid_comma}
\sigma(x) = (\frac{1}{1+e^{-x}}),
\end{equation} 
but this looks ugly, you should use ``\texttt{\textbackslash left (}'' for ``$($'' and ``\texttt{\textbackslash right )}'' for ``$)$'', obtaining
\begin{equation}
\sigma(x) = \left ( \frac{1}{1+e^{-x}} \right ).
\end{equation} 
\item \textbf{Punctuation:} Displayed equations are usually considered to be part of the text and, in turn, they will get the very same punctuation as if they were inline with the text (and part of the sentence). If the sentence ends with a displayed equation, the equation gets a period ``.'' right after it, see Eq.~(\ref{eq:sigmoid}). iI the equation is instead part of a running sentence, which is continued after it, then the equation may be ended by a ``,'' as in Eq.~(\ref{eq:sigmoid_comma}). Use the standard grammar rules and your good sense of flow to assess how equations should be punctuated, I usually read through as if they were plain text.
\end{itemize} 



% !TEX root = template.tex

\section{Results}
\label{sec:results}

In this section, you should provide the numerical results. You are free to decide the structure of this section. As a general ``rule of thumb'', use plots to describe your results, showing, e.g., precision, recall and \mbox{F-measure} as a function of the system (learning) parameters. You can also show the precision matrix. 

\begin{remark}
Present the material in a progressive and logical manner, starting with simple things and adding details and explaining more complex findings as you go. Also, do not try to explain/show multiple concepts within the same sentence. Try to \textbf{address one concept at a time}, explain it properly, and only then move on to the next one.
\end{remark}

\begin{remark}
The best results are obtained by generating the graphs using a vector type file, commonly, either \texttt{encapsulated postscript (eps)} or \texttt{pdf} formats. To plot your figures, use the Latex \texttt{\textbackslash includegraphics} command. Lately, I tend to use pdf more.
\end{remark}

\begin{remark}
If your model has hyper-parameters, show selected results for several values of these. Usually, tables are a good approach to concisely visualize the performance as hyper-parameters change. It is also good to show the results for different flavors of the learning architecture, i.e., how architectural choices affect the overall performance. An example is the use of CNN only or CNN+RNN, or using inception for CNNs, dropout for better generalization or attention models. So you may obtain different models that solve the same problem, e.g., CNN, CNN+RNN, CNN+inception, etc.
\end{remark}

\begin{remark}
Evaluate the tradeoff between the performance of your algorithms and their complexity. This is very important in a real-world situation to decide which algorithm to choose for each specific application. For instance, if you need to create an algorithm for a smartphone you need to prioritize low-complex approaches (battery constraints) thus you probably need to sacrifice accuracy. On the contrary, if you have a server with high computing power you can select the higher-performing algorithm even if the complexity is higher. In your comparison, evaluate at least: accuracy, F1-score, execution time (training/testing), and memory occupation.
\end{remark}

% !TEX root = template.tex

\section{Concluding Remarks}
\label{sec:conclusions}

\red{This section should take max half a page, I personally find it difficult to come up with really useful observations, I mean ones that bring a new contribution with respect to what you have already expounded in the ``Results'' section. In case you have some serious stuff to write, you may also extend the section to 3/4 of a page :-).}\\

In many papers, here you find a summary of what done. It is basically an abstract where instead of using the present tense you use the past participle, as you refer to something that you have already developed in the previous sections. While I did it myself in the past, I now find it rather useless.\\ 

\MR{\textbf{What I would like to see here is:} 
\begin{enumerate}
\item a very short summary of what done, 
\item some (possibly) intelligent observations on the relevance and {\it applicability} of your algorithms / findings, 
\item what is still missing, and can be added in the future to extend your work.\\
\end{enumerate}
The idea is that this section should be {\it useful} and not just a repetition of the abstract (just \mbox{re-phrased} and written using a different tense...).}\\

\red{\textbf{Moreover:} being a project report, I would also like to see a specific paragraph stating 
\begin{enumerate}
\item[4)] what you have learned, and 
\item[5)] any difficulties you may have encountered.
\end{enumerate}}


\section{Exam rules}

What you need to do to pass the exam:
\begin{itemize}
\item Optional: team up with another student. Max. group size is \textbf{two students} per group;
\item Identify a project to work on, devise your own neural network architecture, train and test it on the provided dataset;
\item \textbf{Prepare a written project report} including: i) diagrams, ii) configuration parameters, iii) results with evaluation metrics, iv) your discussion;
\item \textbf{Prior to presenting your work}: upload i) your written report and ii) the code;
\item \textbf{Present your work} using slides (max. duration is 20 minutes): take turns in presenting your work, your individual contribution to the project should clearly emerge. Optional: a final and quick demo with running code is appreciated and will be considered in the calculation of the final grade (see below). 
\end{itemize}

Your final grade will be obtained taking into account the following criteria:
\begin{itemize} 
\item \textbf{Project} (55 points): originality (10 pt.) - data preprocessing techniques (10 pt.) - learning architectures (15 pt.) - comparison against other/existing approaches and performance analysis (accuracy, F1-score, execution time, memory occupation) (10 pt.) - live demo of the code (10 pt.)
\item \textbf{Written report} (35 points): clarity of exposition (10 pt.) - completeness of results (10 pt.) - technical soundness (15 pt.)
\item \textbf{Oral exposition} (20 points): duration (your talk must take max. 20 minutes, using slides) (10 pt.) - clarity of exposition (10 pt.)
\end{itemize}

The final grade will be computed as
\begin{equation}
\textrm{grade} = \max\bigg[\min\Big[0.424242 \times \sum \rm{points} -11.69 ; 32\Big]; 0\bigg]
\end{equation}

\bibliography{biblio}
\bibliographystyle{ieeetr}

\end{document}


